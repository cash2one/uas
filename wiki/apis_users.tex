\part{General:}
\label{general:}

\chapter{1. lookup}
\label{lookup}

lookup user id uri

\section{URL}
\label{url}

\href{https://api.uas.sdo.com/v1/users/lookup}{https:/\slash api.uas.sdo.com\slash v1\slash users\slash lookup}\footnote{\href{https://api.uas.sdo.com/v1/users/lookup}{https:/\slash api.uas.sdo.com\slash v1\slash users\slash lookup}}

\section{Parameters}
\label{parameters}

\begin{table}[htbp]
\begin{minipage}{\linewidth}
\setlength{\tymax}{0.5\linewidth}
\centering
\small
\begin{tabulary}{\textwidth}{@{}LCR@{}} \toprule
Request Parameters&optional&Type&Description\\
\midrule
access\_token&False&String&\\
tel\_number&True&String&User's Telephone Number\\
email&True&String&User's Email\\

\bottomrule

\end{tabulary}
\end{minipage}
\end{table}


\section{Support Format}
\label{supportformat}

JSON

\section{HTTP Request Method}
\label{httprequestmethod}

GET

\section{Authorization}
\label{authorization}

YES

About Authorization \href{http://auth.uas.sdo.com/how_to_auth}{authorization}\footnote{\href{http://auth.uas.sdo.com/how_to_auth}{http:/\slash auth.uas.sdo.com\slash how\_to\_auth}}

\section{Access Limits}
\label{accesslimits}

Level: General
Rates Limits: YES

About Rates Limits \href{http://auth.uas.sdo.com/about_rates}{Rates Limit}\footnote{\href{http://auth.uas.sdo.com/about_rates}{http:/\slash auth.uas.sdo.com\slash about\_rates}}

\section{Response}
\label{response}

Just Response Base Profile 

JSON Example

\begin{verbatim}
{
    count: 1,

    offset: 1,

    limit: 1,

    users:[
    {
        "uid": "87c09c8c-8b82-4e9f-a982-ec577074db4b",

        "tel_number": "13812345678",

        "email": "mawei02@snda.com",

        ......
    }
}
\end{verbatim}


\part{Users:}
\label{users:}

\chapter{1. user}
\label{user}

user CRUD uri

\section{URL}
\label{url}

\href{https://api.uas.sdo.com/v1/user/{uid}}{https:/\slash api.uas.sdo.com\slash v1\slash user\slash \{uid\}}\footnote{\href{https://api.uas.sdo.com/v1/user/{uid}}{https:/\slash api.uas.sdo.com\slash v1\slash user\slash \{uid\}}}

\section{Parameters}
\label{parameters}

\textbar{}--------------------------------------------------------------------\textbar{}\\
\textbar{}Request Parameters \textbar{} Optional \textbar{} Type \textbar{} Description \textbar{}\\
\textbar{}accessToken \textbar{} False \textbar{} String \textbar{} \textbar{}\\
\textbar{}fields \textbar{} True \textbar{} String \textbar{} Fields,support ``base'',``full'',``field1,field2,{\ldots}''\\
\textbar{}command \textbar{} True \textbar{} String \textbar{} C,R,U,D C,D should be in GET, R,U should be in POST 

\section{Support Format}
\label{supportformat}

JSON 

\section{HTTP Request Method}
\label{httprequestmethod}

GET,POST

GET : Retrieve User, Delete User
POST : Create or Update User

\section{Authorization}
\label{authorization}

YES

About Authorization \href{http://auth.uas.sdo.com/how_to_auth}{authorization}\footnote{\href{http://auth.uas.sdo.com/how_to_auth}{http:/\slash auth.uas.sdo.com\slash how\_to\_auth}} 

\section{Access Limits}
\label{accesslimits}

Level: General\\
Rates Limits: YES 

About Rates Limits \href{http://auth.uas.sdo.com/about_rates}{Rates Limit}\footnote{\href{http://auth.uas.sdo.com/about_rates}{http:/\slash auth.uas.sdo.com\slash about\_rates}}

\section{Response}
\label{response}

JSON Example 

\begin{verbatim}
{

    "uid": "87c09c8c-8b82-4e9f-a982-ec577074db4b",

    "name": "Tom",

    "gender": "Male",

    "status":  "user's last status",

    "last_use_date": "20 Apr, 2012",

    ......

    ####end base Profile####

    "blood": "O"  ,

    "marry": "YES"  ,  

    "applications": [
        {
            "app_id": 1,  

            "account_id": 12345,  

            "account_nickname": "maven",  

            "last_status": "just a status",  

            ...   

        }
    ],  

    tel_numbers:[
        {
            "tel_number":  "13512345678",

            "tel_type": "work",

            ......

        },{
            "tel_number":  "13588888888",

            "tel_type": "home",

            ......

        }
    ],

    avatars:[],

    ring_tones:[],

    locations:[],

    ...
}
\end{verbatim}


\chapter{2. user\slash status}
\label{userstatus}

Retrieve User's status Information

User's status List

\section{URL}
\label{url}

\href{https://api.uas.sdo.com/v1/users/{uid}/statues}{https:/\slash api.uas.sdo.com\slash v1\slash user\slash \{uid\slash \}statues}\footnote{\href{https://api.uas.sdo.com/v1/users/{uid}/statues}{https:/\slash api.uas.sdo.com\slash v1\slash users\slash \{uid\}\slash statues}}

\section{Parameters}
\label{parameters}

Request Parameters \textbar{} Optional \textbar{} Type \textbar{} Description\\
accessToken \textbar{} False \textbar{} String\textbar{}\\
app\_id \textbar{} True \textbar{} application id : full, application ids\\
offset \textbar{} True \textbar{} int \textbar{} User's Contacts offset\\
limit \textbar{} True \textbar{} int \textbar{} User's Contacts count you want to load 

offset is the count you retrieve last.
limit is contacts you retrieve once.

\section{Support Format}
\label{supportformat}

JSON

\section{HTTP Request Method}
\label{httprequestmethod}

GET,POST
C
R
U
D

\section{Authorization}
\label{authorization}

YES

About Authorization \href{http://auth.uas.sdo.com/how_to_auth}{authorization}\footnote{\href{http://auth.uas.sdo.com/how_to_auth}{http:/\slash auth.uas.sdo.com\slash how\_to\_auth}}

\section{Access Limits}
\label{accesslimits}

Level: General
Rates Limits: YES

About Rates Limits \href{http://auth.uas.sdo.com/about_rates}{Rates Limit}\footnote{\href{http://auth.uas.sdo.com/about_rates}{http:/\slash auth.uas.sdo.com\slash about\_rates}}

\section{Response}
\label{response}

JSON Example

\begin{verbatim}
{

    "uid": "87c09c8c-8b82-4e9f-a982-ec577074db4b",

    "name": "Tom",  

    "counts": 50,  

    "offset": 1,  

    "limit": 20,  


    "statues": [
            {

                "app_id": 1,

                "status": "status in some application",

                "created_date": "23 Apr 2012 18:06:00",

                ......

            },
            {

                "app_id": 2,

                "status": "status in second application",

                "created_date": "23 Apr 2012 18:08:00",

                ......

            }
        ......
    ]
}
\end{verbatim}


\chapter{3. user\slash key}
\label{userkey}

Retrieve User's application-store-key value

User's application-store-key values

\section{URL}
\label{url}

[https:/\slash api.uas.sdo.com\slash v1\slash user\slash \{uid\}\slash key\slash \{key\}](https:/\slash api.uas.sdo.com\slash v1\slash users\slash \{uid\}\slash key\slash \{key\} users.key")

\section{Parameters}
\label{parameters}

Request Parameters \textbar{} Optional \textbar{} Type \textbar{} Description\\
accessToken \textbar{} False \textbar{} String\textbar{}\\
app\_id \textbar{} True \textbar{} application id : full, application ids\\
offset \textbar{} True \textbar{} int \textbar{} User's Contacts offset\\
limit \textbar{} True \textbar{} int \textbar{} User's Contacts count you want to load 

offset is the count you retrieve last.
limit is contacts you retrieve once.

\section{Support Format}
\label{supportformat}

JSON

\section{HTTP Request Method}
\label{httprequestmethod}

GET,POST
C
R
U
D

\section{Authorization}
\label{authorization}

YES

About Authorization \href{http://auth.uas.sdo.com/how_to_auth}{authorization}\footnote{\href{http://auth.uas.sdo.com/how_to_auth}{http:/\slash auth.uas.sdo.com\slash how\_to\_auth}}

\section{Access Limits}
\label{accesslimits}

Level: General
Rates Limits: YES

About Rates Limits \href{http://auth.uas.sdo.com/about_rates}{Rates Limit}\footnote{\href{http://auth.uas.sdo.com/about_rates}{http:/\slash auth.uas.sdo.com\slash about\_rates}}

\section{Response}
\label{response}

JSON Example

\begin{verbatim}
{

    "uid": "87c09c8c-8b82-4e9f-a982-ec577074db4b",

    "name": "Tom",  

    "counts": 50,  

    "offset": 1,  

    "limit": 20,  

    "key": "stored-key",

    "stored-values": [
            {

                "app_id": 1,

                "value": "application self stored value in some application",

                "created_date": "23 Apr 2012 18:06:00",

                ......

            },
            {

                "app_id": 2,

                "value": "application self stored value in second application",

                "created_date": "23 Apr 2012 18:08:00",

                ......

            }
        ......
    ]
}
\end{verbatim}


\chapter{4. user\slash contacts}
\label{usercontacts}

Retrieve User's Contacts Information

User's Contacts List 

\section{URL}
\label{url}

\href{https://api.uas.sdo.com/v1/users/{uid}/contacts}{https:/\slash api.uas.sdo.com\slash v1\slash user\slash \{uid\slash \}contacts}\footnote{\href{https://api.uas.sdo.com/v1/users/{uid}/contacts}{https:/\slash api.uas.sdo.com\slash v1\slash users\slash \{uid\}\slash contacts}}

\section{Parameters}
\label{parameters}

Request Parameters \textbar{} Optional \textbar{} Type \textbar{} Description\\
accessToken \textbar{} False \textbar{} String\textbar{}\\
fields \textbar{} True \textbar{} user profile type: base, full, fields\\
offset \textbar{} True \textbar{} int \textbar{} User's Contacts offset\\
limit \textbar{} True \textbar{} int \textbar{} User's Contacts count you want to load 

telNumber or email must be has one.
offset is the count you retrieve last.
limit is contacts you retrieve once.
Just like ``select * from contacts limit offset, limit''

\section{Support Format}
\label{supportformat}

JSON 

\section{HTTP Request Method}
\label{httprequestmethod}

GET

\section{Authorization}
\label{authorization}

YES

About Authorization \href{http://auth.uas.sdo.com/how_to_auth}{authorization}\footnote{\href{http://auth.uas.sdo.com/how_to_auth}{http:/\slash auth.uas.sdo.com\slash how\_to\_auth}} 

\section{Access Limits}
\label{accesslimits}

Level: General\\
Rates Limits: YES 

About Rates Limits \href{http://auth.uas.sdo.com/about_rates}{Rates Limit}\footnote{\href{http://auth.uas.sdo.com/about_rates}{http:/\slash auth.uas.sdo.com\slash about\_rates}}

\section{Response}
\label{response}

JSON Example 

\begin{verbatim}
{

    "uid": "87c09c8c-8b82-4e9f-a982-ec577074db4b",

    "name": "Tom",  

    "counts": 50,  

    "offset": 1,  

    "limit": 20,  

    "Contacts": [
            {

                "uid": "87c09c8c-8b82-4e9f-a982-ec577074db4b",

                "name": "Tom",

                "gender": "Male",

                "status":  "user's last status",

                "last_use_date": "20 Apr, 2012",

                ......

                ####end base Profile####

                "blood": "O"  ,

                "marry": "YES"  ,

                tel_numbers:[
                    {
                        "tel_number":  "13512345678",

                        "tel_type": "work",

                        ......

                    },{
                        "tel_number":  "13588888888",

                        "tel_type": "home",

                        ......

                    }
                ],

                avatars:[],

                ring_tones:[],

                locations:[],

                ...
            },
            {

                "uid": "87c09c8c-8b82-4e9f-a982-ec577074db4b",

                "name": "Tom",

                "gender": "Male",

                "status":  "user's last status",

                "last_use_date": "20 Apr, 2012",

                ......

                ####end base Profile####

                "blood": "O"  ,

                "marry": "YES"  ,

                tel_numbers:[
                    {
                        "tel_number":  "13512345678",

                        "tel_type": "work",

                        ......

                    },{
                        "tel_number":  "13588888888",

                        "tel_type": "home",

                        ......

                    }
                ],

                avatars:[],

                ring_tones:[],

                locations:[],

                ...
            },
        ......                  
    ]
}
\end{verbatim}


\chapter{5. user\slash friends}
\label{userfriends}

friends is relative via backend computing.

Retrieve User's friends Information

User's friends List

\section{URL}
\label{url}

\href{https://api.uas.sdo.com/v1/users/{uid}/friends}{https:/\slash api.uas.sdo.com\slash v1\slash user\slash \{uid\}\slash friends}\footnote{\href{https://api.uas.sdo.com/v1/users/{uid}/friends}{https:/\slash api.uas.sdo.com\slash v1\slash users\slash \{uid\}\slash friends}}

\section{Parameters}
\label{parameters}

Request Parameters \textbar{} Optional \textbar{} Type \textbar{} Description\\
accessToken \textbar{} False \textbar{} String\textbar{}\\
fields \textbar{} True \textbar{} user profile type: base, full, fields\\
offset \textbar{} True \textbar{} int \textbar{} User's Contacts offset\\
limit \textbar{} True \textbar{} int \textbar{} User's Contacts count you want to load 

fields is full or fields(``name,tel\_number'')
offset is friends list offset
limit is friends you retrieve once.
Just like ``select * from friends limit offset, limit''

\section{Support Format}
\label{supportformat}

JSON

\section{HTTP Request Method}
\label{httprequestmethod}

GET

\section{Authorization}
\label{authorization}

YES

About Authorization \href{http://auth.uas.sdo.com/how_to_auth}{authorization}\footnote{\href{http://auth.uas.sdo.com/how_to_auth}{http:/\slash auth.uas.sdo.com\slash how\_to\_auth}}

\section{Access Limits}
\label{accesslimits}

Level: General
Rates Limits: YES

About Rates Limits \href{http://auth.uas.sdo.com/about_rates}{Rates Limit}\footnote{\href{http://auth.uas.sdo.com/about_rates}{http:/\slash auth.uas.sdo.com\slash about\_rates}}

\section{Response}
\label{response}

JSON Example

\begin{verbatim}
{

    "uid": "87c09c8c-8b82-4e9f-a982-ec577074db4b",

    "name": "Tom",

    "counts": 50,  

    "offset": 1,  

    "limit": 20,

    "Contacts": [
            {

                "uid": "87c09c8c-8b82-4e9f-a982-ec577074db4b",

                "name": "Tom",

                "gender": "Male",

                "status":  "user's last status",

                "last_use_date": "20 Apr, 2012",

                ......

                ####end base Profile####

                "blood": "O"  ,

                "marry": "YES"  ,

                tel_numbers:[
                    {
                        "tel_number":  "13512345678",

                        "tel_type": "work",

                        ......

                    },{
                        "tel_number":  "13588888888",

                        "tel_type": "home",

                        ......

                    }
                ],

                avatars:[],

                ring_tones:[],

                locations:[],

                ...
            },
            {

                "uid": "87c09c8c-8b82-4e9f-a982-ec577074db4b",

                "name": "Tom",

                "gender": "Male",

                "status":  "user's last status",

                "last_use_date": "20 Apr, 2012",

                ......

                ####end base Profile####

                "blood": "O"  ,

                "marry": "YES"  ,

                tel_numbers:[
                    {
                        "tel_number":  "13512345678",

                        "tel_type": "work",

                        ......

                    },{
                        "tel_number":  "13588888888",

                        "tel_type": "home",

                        ......

                    }
                ],

                avatars:[],

                ring_tones:[],

                locations:[],

                ...
            },
        ......
    ]
}
\end{verbatim}


\chapter{6. user\slash in\_contacts}
\label{userin_contacts}

Retrieve User's friends Information

User's friends List

\section{URL}
\label{url}

\href{https://api.uas.sdo.com/v1/users/{uid}/in_contacts}{https:/\slash api.uas.sdo.com\slash v1\slash user\slash \{uid\}\slash in\_contacts}\footnote{\href{https://api.uas.sdo.com/v1/users/{uid}/in_contacts}{https:/\slash api.uas.sdo.com\slash v1\slash users\slash \{uid\}\slash in\_contacts}}

\section{Parameters}
\label{parameters}

Request Parameters \textbar{} Optional \textbar{} Type \textbar{} Description\\
accessToken \textbar{} False \textbar{} String\textbar{}\\
fields \textbar{} True \textbar{} user profile type: base, full, fields\\
offset \textbar{} True \textbar{} int \textbar{} User's Contacts offset\\
limit \textbar{} True \textbar{} int \textbar{} User's Contacts count you want to load 

fields is full or fields(``name,tel\_number'')
offset is friends list offset
limit is friends you retrieve once.
Just like ``select * from friends limit offset, limit''

\section{Support Format}
\label{supportformat}

JSON

\section{HTTP Request Method}
\label{httprequestmethod}

GET

\section{Authorization}
\label{authorization}

YES

About Authorization \href{http://auth.uas.sdo.com/how_to_auth}{authorization}\footnote{\href{http://auth.uas.sdo.com/how_to_auth}{http:/\slash auth.uas.sdo.com\slash how\_to\_auth}}

\section{Access Limits}
\label{accesslimits}

Level: General
Rates Limits: YES

About Rates Limits \href{http://auth.uas.sdo.com/about_rates}{Rates Limit}\footnote{\href{http://auth.uas.sdo.com/about_rates}{http:/\slash auth.uas.sdo.com\slash about\_rates}}

\section{Response}
\label{response}

JSON Example

\begin{verbatim}
{

    "uid": "87c09c8c-8b82-4e9f-a982-ec577074db4b",

    "name": "Tom",  

    "counts": 50,  

    "offset": 1,  

    "limit": 20,  

    "in_contacts": [
            {

                "uid": "87c09c8c-8b82-4e9f-a982-ec577074db4b",

                "name": "Tom",

                "gender": "Male",

                "status":  "user's last status",

                "last_use_date": "20 Apr, 2012",

                ......

                ####end base Profile####

                "blood": "O"  ,

                "marry": "YES"  ,

                tel_numbers:[
                    {
                        "tel_number":  "13512345678",

                        "tel_type": "work",

                        ......

                    },{
                        "tel_number":  "13588888888",

                        "tel_type": "home",

                        ......

                    }
                ],

                avatars:[],

                ring_tones:[],

                locations:[],

                ...
            },
            {

                "uid": "87c09c8c-8b82-4e9f-a982-ec577074db4b",

                "name": "Tom",

                "gender": "Male",

                "status":  "user's last status",

                "last_use_date": "20 Apr, 2012",

                ......

                ####end base Profile####

                "blood": "O"  ,

                "marry": "YES"  ,

                tel_numbers:[
                    {
                        "tel_number":  "13512345678",

                        "tel_type": "work",

                        ......

                    },{
                        "tel_number":  "13588888888",

                        "tel_type": "home",

                        ......

                    }
                ],

                avatars:[],

                ring_tones:[],

                locations:[],

                ...
            },
        ......
    ]
}
\end{verbatim}


\chapter{7. user\slash contacts\slash app}
\label{usercontactsapp}

Retrieve contacts is used the app list

\section{URL}
\label{url}

\href{https://api.uas.sdo.com/v1/user/{uid}/contacts/app/{app_id}}{https:/\slash api.uas.sdo.com\slash v1\slash user\slash \{uid\}\slash contacts\slash app\slash \{app\_id\}}\footnote{\href{https://api.uas.sdo.com/v1/user/{uid}/contacts/app/{app_id}}{https:/\slash api.uas.sdo.com\slash v1\slash user\slash \{uid\}\slash contacts\slash app\slash \{app\_id\}}}

\section{Parameters}
\label{parameters}

\textbar{}Request Parameters \textbar{} Optional \textbar{} Type \textbar{} Description\\
\textbar{}accessToken \textbar{} False \textbar{} String\textbar{}\\
\textbar{}uid \textbar{} False \textbar{} String\textbar{} User's id\\
\textbar{}app\_id \textbar{} False \textbar{} String\textbar{} support full, app id array such as ``1,2,3''\\
offset \textbar{} True \textbar{} int \textbar{} User's Contacts offset\\
limit \textbar{} True \textbar{} int \textbar{} User's Contacts count you want to load 

\section{Support Format}
\label{supportformat}

JSON

\section{HTTP Request Method}
\label{httprequestmethod}

GET

\section{Authorization}
\label{authorization}

YES

About Authorization \href{http://auth.uas.sdo.com/how_to_auth}{authorization}\footnote{\href{http://auth.uas.sdo.com/how_to_auth}{http:/\slash auth.uas.sdo.com\slash how\_to\_auth}}

\section{Access Limits}
\label{accesslimits}

Level: General
Rates Limits: YES

About Rates Limits \href{http://auth.uas.sdo.com/about_rates}{Rates Limit}\footnote{\href{http://auth.uas.sdo.com/about_rates}{http:/\slash auth.uas.sdo.com\slash about\_rates}}

\section{Response}
\label{response}

JSON Example

\begin{verbatim}
{
    "app_id": "full",

    "app_name": "Application name",  

    "counts": 50,  

    "offset": 1,  

    "limit": 20,

    [{
        "uid": "87c09c8c-8b82-4e9f-a982-ec577074db4b",

        accounts:[

            {
                "app_id": 1,

                "user_id_in_app": "12345",

                "status":  "user's last status",

                "last_use_date": "20 Apr, 2012",

                ......
            },{
                "app_id": 9,

                "user_id_in_app": "1356934",

                "status": "user's status in application",

                ......
            }
        ]
        ......
    },{
        "uid": "87c09c8c-8b82-4e9f-a982-ec577074db4c",

        accounts:[

            {
                "app_id": 1,

                "user_id_in_app": "88888",

                "status":  "user's last status",

                "last_use_date": "20 Apr, 2012",

                ......
            }
        ]
        ......
    },
    ......
    ]
}
\end{verbatim}


\chapter{8. user\slash contacts\slash app\slash status}
\label{usercontactsappstatus}

Retrieve contacts last status Information

\section{URL}
\label{url}

\href{https://api.uas.sdo.com/v1/user/{uid}/contacts/app/{app_id}/status}{https:/\slash api.uas.sdo.com\slash v1\slash user\slash \{uid\}\slash contacts\slash app\slash \{app\_id\}\slash status}\footnote{\href{https://api.uas.sdo.com/v1/user/{uid}/contacts/app/{app_id}/status}{https:/\slash api.uas.sdo.com\slash v1\slash user\slash \{uid\}\slash contacts\slash app\slash \{app\_id\}\slash status}}

\section{Parameters}
\label{parameters}

Request Parameters \textbar{} Optional \textbar{} Type \textbar{} Description\\
accessToken \textbar{} False \textbar{} String \textbar{}\\
uid \textbar{} True \textbar{} String \textbar{} User's id\\
app\_id \textbar{} True \textbar{} String \textbar{} Application id, support full,and fields ``1,2,3''\\
status \textbar{} True \textbar{} String \textbar{} if Application allow, Application could support status features without coding.\\
offset \textbar{} True \textbar{} int \textbar{} User's Contacts offset\\
limit \textbar{} True \textbar{} int \textbar{} User's Contacts count you want to load 

\section{Support Format}
\label{supportformat}

JSON

\section{HTTP Request Method}
\label{httprequestmethod}

GET,POST

CRUD

\section{Authorization}
\label{authorization}

YES

About Authorization \href{http://auth.uas.sdo.com/how_to_auth}{authorization}\footnote{\href{http://auth.uas.sdo.com/how_to_auth}{http:/\slash auth.uas.sdo.com\slash how\_to\_auth}}

\section{Access Limits}
\label{accesslimits}

Level: General
Rates Limits: YES

About Rates Limits \href{http://auth.uas.sdo.com/about_rates}{Rates Limit}\footnote{\href{http://auth.uas.sdo.com/about_rates}{http:/\slash auth.uas.sdo.com\slash about\_rates}}

\section{Response}
\label{response}

JSON Example

\begin{verbatim}
{
    "app_id": 1

    "app_name": "Application name",  

    "counts": 50,  

    "offset": 1,  

    "limit": 20,


    [{
        "uid": "87c09c8c-8b82-4e9f-a982-ec577074db4b",

        statuses: [
            {
                "app_id": 12345,

                "status": "I just want to say something!",

                "created_date": "2012-05-01 13:13:33"
            },{
                "app_id": 54321,

                "status": "Someone say something!",

                "created_date": "2012-05-01 13:13:33"
            }

        ]
        ......
    },{
        "uid": "87c09c8c-8b82-4e9f-a982-ec577074db4c",

        statuses: [
            {
                "app_id": 12345,

                "status": "I just want to say something!",

                "created_date": "2012-05-01 13:13:33"
            }
        ]

    },{
        "uid": "87c09c8c-8b82-4e9f-a982-ec577074db4d",

        statuses: [
            {
                "app_id": 12345,

                "status": "I just want to say something!",

                "created_date": "2012-05-01 13:13:33"
            },{
                "app_id": 54321,

                "status": "Someone say something!",

                "created_date": "2012-05-01 13:13:33"
            },{
                "app_id": 333,

                "status": "Someone say something!",

                "created_date": "2012-05-01 13:13:33"
            }
        ]
    },
    ......
    ]
    ......
}
\end{verbatim}


\chapter{9. user\slash contacts\slash app\slash key}
\label{usercontactsappkey}

Retrieve contacts last status Information

\section{URL}
\label{url}

\href{https://api.uas.sdo.com/v1/user/{uid}/contacts/app/{app_id}/key/{key}}{https:/\slash api.uas.sdo.com\slash v1\slash user\slash \{uid\}\slash contacts\slash app\slash \{app\_id\}\slash key\slash \{key\}}\footnote{\href{https://api.uas.sdo.com/v1/user/{uid}/contacts/app/{app_id}/key/{key}}{https:/\slash api.uas.sdo.com\slash v1\slash user\slash \{uid\}\slash contacts\slash app\slash \{app\_id\}\slash key\slash \{key\}}}

\section{Parameters}
\label{parameters}

Request Parameters \textbar{} Optional \textbar{} Type \textbar{} Description\\
accessToken \textbar{} False \textbar{} String \textbar{}\\
uid \textbar{} True \textbar{} String \textbar{} User's id\\
app\_id \textbar{} True \textbar{} String \textbar{} Application id, support full,and fields ``1,2,3''\\
key \textbar{} False \textbar{} String \textbar{} Application want to store meta key\\
value \textbar{} False \textbar{} Object \textbar{} Application want to store meta value\\
offset \textbar{} True \textbar{} int \textbar{} User's Contacts offset\\
limit \textbar{} True \textbar{} int \textbar{} User's Contacts count you want to load 

\section{Support Format}
\label{supportformat}

JSON

\section{HTTP Request Method}
\label{httprequestmethod}

GET,POST

CRUD

\section{Authorization}
\label{authorization}

YES

About Authorization \href{http://auth.uas.sdo.com/how_to_auth}{authorization}\footnote{\href{http://auth.uas.sdo.com/how_to_auth}{http:/\slash auth.uas.sdo.com\slash how\_to\_auth}}

\section{Access Limits}
\label{accesslimits}

Level: General
Rates Limits: YES

About Rates Limits \href{http://auth.uas.sdo.com/about_rates}{Rates Limit}\footnote{\href{http://auth.uas.sdo.com/about_rates}{http:/\slash auth.uas.sdo.com\slash about\_rates}}

\section{Response}
\label{response}

JSON Example

\begin{verbatim}
{
    "app_id": 1

    "app_name": "Application name",  

    "counts": 50,  

    "offset": 1,  

    "limit": 20,


    [{
        "uid": "87c09c8c-8b82-4e9f-a982-ec577074db4b",

        statuses: [
            {
                "app_id": 12345,

                "self-stored-key": "self-stored-value for this app",

                "created_date": "2012-05-01 13:13:33"
            },{
                "app_id": 54321,

                "self-stored-key": "self-stored-value for this app",

                "created_date": "2012-05-01 13:13:33"
            }

        ]
        ......
    },{
        "uid": "87c09c8c-8b82-4e9f-a982-ec577074db4c",

        statuses: [
            {
                "app_id": 12345,

                "self-stored-key": "self-stored-value for this app",

                "created_date": "2012-05-01 13:13:33"
            }
        ]

    },
    ......
    ]
    ......
}
\end{verbatim}


\chapter{10. users batch update}
\label{usersbatchupdate}

Update Users Information

\section{URL}
\label{url}

\href{https://api.uas.sdo.com/v1/users/update}{https:/\slash api.uas.sdo.com\slash v1\slash users\slash update}\footnote{\href{https://api.uas.sdo.com/v1/users/update}{https:/\slash api.uas.sdo.com\slash v1\slash users\slash update}}

\section{Parameters}
\label{parameters}

Request Parameters \textbar{} Optional \textbar{} Type \textbar{} Description\\
accessToken \textbar{} False \textbar{} String \textbar{}\\
userData \textbar{} True \textbar{} String \textbar{} User's Data in JSON Format\\
count \textbar{} int \textbar{} int \textbar{} how many users you batch update . 

\section{Support Format}
\label{supportformat}

JSON

\section{HTTP Request Method}
\label{httprequestmethod}

POST

\section{Authorization}
\label{authorization}

YES

About Authorization \href{http://auth.uas.sdo.com/how_to_auth}{authorization}\footnote{\href{http://auth.uas.sdo.com/how_to_auth}{http:/\slash auth.uas.sdo.com\slash how\_to\_auth}}

\section{Access Limits}
\label{accesslimits}

Level: General
Rates Limits: YES

About Rates Limits \href{http://auth.uas.sdo.com/about_rates}{Rates Limit}\footnote{\href{http://auth.uas.sdo.com/about_rates}{http:/\slash auth.uas.sdo.com\slash about\_rates}}

\section{Response}
\label{response}

JSON Example

\begin{verbatim}
{

    "uid": "87c09c8c-8b82-4e9f-a982-ec577074db4b",

    "name": "Tom",

    "gender": "Male",

    ......

    ####end base Profile####

    "blood": "O"  ,

    "marry": "YES"  ,

    tel_numbers:[
        {
            "tel_number":  "13512345678",

            "tel_type": "work",

            ......

        },{
            "tel_number":  "13588888888",

            "tel_type": "home",

            ......

        }
    ],

    avatars:[],

    ring_tones:[],

    locations:[],

    ...
}
\end{verbatim}


\part{Application:}
\label{application:}

\begin{verbatim}
Empty Now.  
\end{verbatim}


\part{Errors:}
\label{errors:}

\chapter{1. Uniform Error Response}
\label{uniformerrorresponse}

\section{Response}
\label{response}

JSON Example

\begin{verbatim}
{
    "error":  {

        "message": "(#401) Error Message",

        "type": "Exception Type",

        "code": 401
    }
}
\end{verbatim}


Exceptions:

\textbar{} --------------------------------\textbar{}\\
\textbar{} code \textbar{} message \textbar{} type \textbar{}\\
\textbar{} 401 \textbar{} Forbidden \textbar{} Auth \textbar{}\\
\textbar{} 701 \textbar{} Over Limits\textbar{} Rates \textbar{}\\
{\ldots}
